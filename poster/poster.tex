\documentclass{beamer}
\usepackage[orientation=portrait,size=a0,scale=1.4,debug]{beamerposter}
\mode<presentation>{\usetheme{ZH}}
\usepackage{fontspec}
\usepackage{microtype}
\usepackage[english]{babel} % required for rendering German special characters
\usepackage{siunitx} %pretty measurement unit rendering
\usepackage{hyperref} %enable hyperlink for urls
\usepackage{ragged2e}
\usepackage[font=scriptsize,justification=justified]{caption}
\usepackage{array,booktabs,tabularx}
\usepackage{float}
\usepackage{subfig}

\newcommand{\sgn}{\operatorname{sgn}}
\newcommand{\conv}{\operatorname{conv}}
\newcommand{\vect}{\operatorname{vect}}
\DeclareMathOperator*{\argmin}{arg\,min}
\DeclareMathOperator*{\argmax}{arg\,max}
\newcommand*\diff{\mathop{}\!\mathrm{d}}
\newcommand{\Var}{\mathrm{Var}}
\newcommand*\ri{\mathop{}\!\mathrm{ri}}
\newcommand*\aff{\mathop{}\!\mathrm{aff}}
\newcommand*\dom{\mathop{}\!\mathrm{dom}}
\newcommand*\epi{\mathop{}\!\mathrm{epi}}
\newcommand*\diag{\mathop{}\!\mathrm{diag}}
\newcommand*\cov{\mathop{}\!\mathrm{cov}}
\newcommand*\var{\mathop{}\!\mathrm{var}}
\newcommand*\corr{\mathop{}\!\mathrm{corr}}

\newcolumntype{Z}{>{\centering\arraybackslash}X} % centered tabularx columns
\sisetup{per=frac,fraction=sfrac}

\title{Semi-parametric blind sources separation with \huge Kernel-ICA}
\author{Guillaume Ausset}
\institute[MASH]{Université Paris Dauphine}
\date{January 4, 2017}

% edit this depending on how tall your header is. We should make this scaling automatic :-/
\newlength{\columnheight}
\setlength{\columnheight}{104cm}

\begin{document}
\begin{frame}
\begin{columns}
	\begin{column}{.5\textwidth}
		\begin{beamercolorbox}[center]{postercolumn}
			\begin{minipage}{.98\textwidth}  % tweaks the width, makes a new \textwidth
				\parbox[t][\columnheight]{\textwidth}{ % must be some better way to set the the height, width and textwidth simultaneously
					\begin{myblock}{Introduction}
					It is rare for the quantity of interest and the quantity measured to perfectly match often our signal of interest $s \in \mathcal{R}^{m \times n}$ is actually mixed with itself. We will set up a certain number of recording devices, generally $n$ and from our recorded signals try to recover the sources. If we assume the sources are linearly mixed our problem is then:

						\begin{align*}
							x = A s
						\end{align*}

						where $x \in \mathcal{R}^{k \times n}$, $A \in \mathcal{R}^{k \times m}$. While not always true or needed we will assume that $A$ is square and invertible and call $A^{-1} = W$.

						We will see that the sole hypothesis of the independence of the $s_i$ is enough to find a solution to the problem.

						Because of the way the problem is posed one cannot hope to recover the real $s$ as for example a scaling or permutation of $A$ leaves the problem unchanged and we therefore cannot hope to recover the magnitudes (and therefore signs) of the $s_i$ or their order.
					\end{myblock}\vfill
					\begin{myblock}{ICA as an eigenvalue problem}
						We want to optimize a measure of independence of our estimated sources. Most approaches try to optimize an approximation of the mutual information. The measure chosen here \cite{Bach2002} is for two variables:
						\begin{align*}
							\rho_\mathcal{F} = \max_{f_1, f_2 \in \mathcal{F}} \corr (f_1(x_1), f_2 (x_2))
						\end{align*}
						By exploiting the kernel trick we can obtain
						\begin{align*}
							\rho_\mathcal{F} = \max_{f_1, f_2 \in \mathcal{F}} \corr ( \langle \Phi_1 (x_1), f_1 \rangle , \langle \Phi_2 (x_2), f_2 \rangle)
						\end{align*}
						We recognize a CCA problem and if we note $K_1$ and $K_2$ the respective Gram matrices (assuming they are centred) we get the problem:
						\begin{align*}
							\begin{pmatrix}
								0 & K_1 K_2 \\
								K_2 K_1 & 0
							\end{pmatrix} \begin{pmatrix}
								\alpha_1 \\ \alpha_2
							\end{pmatrix}
							= \rho \begin{pmatrix}
								K_1^2 & 0 \\
								0 & K_2^2
							\end{pmatrix} \begin{pmatrix}
								\alpha_1 \\ \alpha_2
							\end{pmatrix}
						\end{align*}
						This problem is unfortunately not well posed and will always be equal to $0$ for most kernels, we therefore adopt a regularized version as an estimator:
						\begin{align*}
							\rho_\mathcal{F} = \max_{f_1, f_2 \in \mathcal{F}} \frac{\cov (f_1(x_1), f_2 (x_2))}{(\var f_1 (x_1) + \kappa \lVert f_1 \rVert^2 _\mathcal{F} )^{1/2} (\var f_2 (x_2) + \kappa \lVert f_2 \rVert^2 _\mathcal{F} )^{1/2}}
						\end{align*}
						The problem estimated at the first order is then:
						\begin{align*}
							\begin{pmatrix}
								0 & K_1 K_2 \\
								K_2 K_1 & 0
							\end{pmatrix} \begin{pmatrix}
								\alpha_1 \\ \alpha_2
							\end{pmatrix}
							= \rho \begin{pmatrix}
								(K_1+\frac{N \kappa}{2} \mathbb{I})^2 & 0 \\
								0 & (K_2+\frac{N \kappa}{2} \mathbb{I})^2
							\end{pmatrix} \begin{pmatrix}
								\alpha_1 \\ \alpha_2
							\end{pmatrix}
						\end{align*}
						Of course we are interested in solving the $m$-variables problems, we can easily extend CCA to $m$ variables.
						%\begin{align*}
						%	&\begin{pmatrix}
						%		0 & K_1 K_2  & \cdots & K_1 K_m\\
						%		K_2 K_1 & 0 & \cdots & K_2 K_m \\
						%		\vdots & \vdots & \ddots & \vdots \\
						%		K_m K_1 & K_m K_2 & \cdots & 0
						%	\end{pmatrix} \begin{pmatrix}
						%		\alpha_1 \\ \alpha_2 \\ \vdots \\ \alpha_m
						%	\end{pmatrix} \\
						%	&= \rho \begin{pmatrix}
						%		(K_1+\frac{N \kappa}{2} \mathbb{I})^2 & 0  & \cdots & 0\\
						%		0 & (K_2+\frac{N \kappa}{2} \mathbb{I})^2 & \cdots & 0 \\
						%		\vdots & \vdots & \ddots & \vdots \\
						%		0 & 0 & \cdots & (K_m+\frac{N \kappa}{2} \mathbb{I})^2
						%	\end{pmatrix} \begin{pmatrix}
						%		\alpha_1 \\ \alpha_2 \\ \vdots \\ \alpha_m
						%	\end{pmatrix}
						%\end{align*}
						We can then transform that problem in the problem of finding the eigenvalues of:
						\begin{align*}
							&\tilde{\mathcal{K}}_\kappa = \begin{pmatrix}
								\mathbb{I} & r_\kappa (K_1)  r_\kappa (K_2)  & \cdots &  r_\kappa (K_1) r_\kappa (K_m)\\
								r_\kappa (K_2)  r_\kappa (K_1) & \mathbb{I} & \cdots & r_\kappa (K_2) r_\kappa (K_m) \\
								\vdots & \vdots & \ddots & \vdots \\
								r_\kappa (K_m)  r_\kappa (K_1) & r_\kappa (K_m) r_\kappa (K_2) & \cdots & \mathbb{I}
							\end{pmatrix} \\
							& r_\kappa (K_i) = K_i (K_i + \frac{N \kappa}{2} \mathbb{I})^{-1}
						\end{align*}
						We then optimize:
						\begin{align*}
							J(W) = -\frac{1}{2} \det \tilde{\mathcal{K}}_\kappa
						\end{align*}
					\end{myblock}\vfill
					\begin{myblock}{Reducing Complexity}
						If we decompose the $K_i$ as
						\begin{align*}
							K_i = G_i G_i^\intercal = U_i \Lambda_i U_i^\intercal
						\end{align*}
						with $\Lambda_i$ diagonal then if $R_i$ is $\Lambda_i$ regularized by $\lambda \to \frac{\lambda}{\lambda + N \kappa / 2}$ we have
						\begin{align*}
							\tilde{\mathcal{K}}_\kappa = (\mathcal{U} \mathcal{V}) \begin{pmatrix}
							\mathcal{R}_\kappa & 0 \\
							0 & \mathbb{I}
							\end{pmatrix} (\mathcal{U} \mathcal{V})^\intercal
						\end{align*}
						with
						\begin{align*}
							\mathcal{R}_\kappa =  \begin{pmatrix}
								\mathbb{I} & R_1 U_1^\intercal U_2 R_2  & \cdots &  R_1 U_1^\intercal U_m R_m \\
								 R_2 U_2^\intercal U_1 R_1 & \mathbb{I} & \cdots & R_2 U_2^\intercal U_m R_m \\
								\vdots & \vdots & \ddots & \vdots \\
								R_m U_m^\intercal U_1 R_1 & R_m U_m^\intercal U_2 R_2 & \cdots & \mathbb{I} \\
							\end{pmatrix}
						\end{align*}
						And therefore
						\begin{align*}
							\det \tilde{\mathcal{K}}_\kappa = \det \mathcal{R}_\kappa
						\end{align*}
						We will therefore make heavy use of incomplete Choleski decomposition to reduce the computational complexity.
					\end{myblock}\vfill
		}\end{minipage}\end{beamercolorbox}
	\end{column}
	\begin{column}{.5\textwidth}
		\begin{beamercolorbox}[center]{postercolumn}
			\begin{minipage}{.98\textwidth} % tweaks the width, makes a new \textwidth
				\parbox[t][\columnheight]{\textwidth}{ % must be some better way to set the the height, width and textwidth simultaneously
					\begin{myblock}{Optimization on the Stiefel Manifold}
						Our problem is
						\begin{align*}
							&\min_W J(W) \\
							\text{s.t } &W W^\intercal = \mathbb{I}
						\end{align*}
						The set $\{ W \mid W W^\intercal = \mathbb{I} \}$ has a particular geometry: it is Riemannian manifolds and most common optimization procedures can be performed on it \cite{Edelman1998}.
						The simplest implementation is steepest descent along geodesics in the direction of the gradient using the following: if $W, H \in \mathcal{R}^{m \times n}$ s.t $W^\intercal W = \mathbb{I}$ and $A = W^\intercal H$ skew-symmetric then the geodesic on the Stiefel manifold emanating from $W$ in direction $H$ is given by the curve
						\begin{align*}
							& &W (t) = W M (t) + Q N (t) \\
							&\text{where} & QR = (\mathbb{I} - W W^\intercal) H \\
							&\text{and} &\begin{pmatrix}
							 M(t) \\ N(t)
						 \end{pmatrix} = \exp \left( t \begin{pmatrix}
							 A & - R^\intercal \\
							 R & 0
							\end{pmatrix} \begin{pmatrix}
							 \mathbb{I}_n \\ 0
						 \end{pmatrix}\right)
						\end{align*}
						Given the cost of the gradient computations a natural extension is to perform conjugate gradient on the manifold, see \cite{Edelman1998} for the procedure.
					\end{myblock}\vfill
					\begin{myblock}{Unmixing images}
						\begin{figure}[H]
						  \centering
							\subfloat{\includegraphics[width=0.20\textwidth]{img/lena}}
						  \hfill
							\subfloat{\includegraphics[width=0.20\textwidth]{img/fabio}}
						  \hfill
							\subfloat{\includegraphics[width=0.20\textwidth]{img/elaine}}
						  \caption{Original images}
						\end{figure}
						\begin{figure}[H]
						  \centering
							\subfloat{\includegraphics[width=0.20\textwidth]{img/mix1}}
						  \hfill
							\subfloat{\includegraphics[width=0.20\textwidth]{img/mix2}}
						  \hfill
							\subfloat{\includegraphics[width=0.20\textwidth]{img/mix3}}
						  \caption{Mixed images}
						\end{figure}
						\begin{figure}[H]
						  \centering
							\subfloat{\includegraphics[width=0.20\textwidth]{img/s1}}
						  \hfill
							\subfloat{\includegraphics[width=0.20\textwidth]{img/s2}}
						  \hfill
							\subfloat{\includegraphics[width=0.20\textwidth]{img/s3}}
						  \caption{Retrieved images}
						\end{figure}
					\end{myblock}\vfill
					\begin{myblock}{Independent images basis}
						This time the quantity of interest is $W$ the unmixing matrix, $s$ representing the random coefficients in the ICA basis of natural images. In this setting the columns of $A$ form the basis and the columns of $W$ the detectors.
						\begin{figure}[H]
							\centering
							\subfloat{\includegraphics[width=0.9\textwidth]{img/basis}}
							\caption{ICA basis of the $3$ previous images}
						\end{figure}
					\end{myblock}\vfill
					\begin{myblock}{References}
						\footnotesize
						\bibliographystyle{abbrv}
						\bibliography{./Biblio}
					\end{myblock}\vfill
		}\end{minipage}\end{beamercolorbox}
	\end{column}
\end{columns}
\end{frame}
\end{document}
